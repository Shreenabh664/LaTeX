\documentclass[12pt]{article}
\usepackage[utf8]{inputenc}
\usepackage{xcolor}
\usepackage[T1]{fontenc}
\usepackage{pagecolor}
\usepackage{amssymb}
\usepackage{lmodern}
\usepackage{mathtools, nccmath}
\usepackage{courier}
\usepackage[overload]{empheq}
\usepackage[inline, shortlabels]{enumitem}
\usepackage{amsmath}
\usepackage{mathtools} 
\definecolor{myyellow}{RGB}{225,225,100}
\definecolor{myred}{RGB}{220,100,100}
\definecolor{mygreen}{RGB}{120,225,120}
\definecolor{myblue}{RGB}{100,200,255}
\definecolor{mypurple}{RGB}{200,140,255}
\definecolor{myorange}{RGB}{255,150,50}
\color{white}
\pagecolor{black}
\title{Solution to Linear algebra \#1}
\author{@all.about.mathematics}
\date{}

\begin{document}
\maketitle
\large
\section{Problem}
Let $A$ be a square matrix. Then reflect its entries along the diagonal other than the diagonal and let the new matrix be $B$. Prove that $\det(A)=\det(B)$. 

\medskip
\noindent
For example, if
$$
A=
\begin{bmatrix} 
1 & 4 & 7 \\
2 & 5 & 8 \\
3 & 6 & 9 \\
\end{bmatrix}
\implies
B=
\begin{bmatrix} 
9 & 8 & 7 \\
6 & 5 & 4 \\
3 & 2 & 1 \\
\end{bmatrix}
$$
\newpage
\section{Solution}
Let's use the example provided in the last page. First, let's take the transpose of A, which does not change its determinant.
$$A=
\begin{bmatrix} 
1 & 4 & 7 \\
2 & 5 & 8 \\
3 & 6 & 9 \\
\end{bmatrix}
\implies
A^T=
\begin{bmatrix} 
1 & 2 & 3 \\
4 & 5 & 6 \\
7 & 8 & 9 \\
\end{bmatrix}$$
We can see that $\mathbf{r_1}\longleftrightarrow \mathbf{r_3}$ and $\mathbf{c_1} \longleftrightarrow \mathbf{c_3} $ turns $A^T$ into $B$.
\medskip

\noindent{Similarly, we can deduce that for a $n\times n$ matrix $A$, the following EROs and ECOs can turn $A^T$ into it's corresponding $B$:}
$$\{\mathbf{r_1}\longleftrightarrow\mathbf{r_n}\:,\: \mathbf{r_2}\longleftrightarrow\mathbf{r_{n-1}}\cdots \mathbf{r_{\left \lfloor{n/2}\right \rfloor}}\longleftrightarrow \mathbf{r_{\left \lceil{n/2}\right \rceil}}\} $$
$$\{\mathbf{c_1}\longleftrightarrow\mathbf{c_n}\:,\: \mathbf{c_2}\longleftrightarrow\mathbf{c_{n-1}}\cdots \mathbf{c_{\left \lfloor{n/2}\right \rfloor}}\longleftrightarrow \mathbf{c_{\left \lceil{n/2}\right \rceil}}\}$$
It's trivial that both sets have the same number of operations. Let that number be $m$. Since these Type I operations multiply the determinant by $-1$, $\det(B)$ can be calculated like this.
$$\det(B)=\det(A^T)(-1)^m(-1)^m= \det(A)(-1)^{2m}= \det(A)$$
Therefore our claim is proved.

\end{document}

