\documentclass{article}
\usepackage[utf8]{inputenc}
\usepackage{amsmath}
\usepackage{amssymb}
\usepackage[legalpaper, portrait, margin=1in]{geometry}

\title{Calculus Q.2}
\author{Shreenabh Agrawal}
\date{\today}

\begin{document}

\maketitle

\section{Question}

Evaluate the Integral:
$$\int\limits_{0}^{\infty} \frac{\left(x^{2}-1\right) \ln x}{1+x^{6}} \: d x$$
\section{Solution}

Taking Substitution,
$$\begin{aligned}
x &=\tan ^{1 / 3} \theta \\
d x &=\frac{1}{3} \tan ^{-2 / 3} \theta \sec ^{2} \theta \: d \theta
\end{aligned}$$
The Integral Becomes,

$$\begin{aligned}
{\Rightarrow} I &=\frac{1}{3} \int\limits_{0}^{\pi / 2} \frac{\left(\tan ^{2 / 3} \theta-1\right) \ln (\tan \theta)}{1+\tan ^{2} \theta}\left(\frac{\tan ^{-2 / 3} \theta \sec ^{2} \theta \: d \theta}{3}\right) \\
&=\frac{1}{9} \int\limits_{0}^{\pi / 2}\left(1-\tan ^{-2 / 3} \theta\right) \ln \tan \theta \: d \theta
\end{aligned}$$
We know the Formula,

$$\int\limits_{0}^{\pi / 2} \tan ^{n}(x) \log (\tan (x)) d x=\pi^{2} \sin ^{3}\left(\frac{\pi n}{2}\right) \csc ^{2}(\pi n)$$

\begin{flushright}
[For $-1<n<1$]
\end{flushright}
Using this, and Splitting the Integral,

$$\begin{aligned}
\Rightarrow I  
&= \frac{1}{9}\left[\int\limits_{0}^{\pi / 2} \ln (\tan \theta) \: d \theta-\int\limits_{0}^{\pi / 2}\left(\tan ^{-2 / 3} \theta\right) \ln (\tan \theta) \: d \theta\right]\\
&=\frac{1}{9}\left[0-\pi^{2} \sin ^{3}\left(\frac{\pi}{3}\right) \csc ^{2}\left(\frac{\pi}{3}\right)\right]
\end{aligned}$$

Thus, the final answer is:
$$\boxed{I=\frac{\pi^{2}}{6 \sqrt{3}}}$$

\end{document}
