\documentclass{article}
\usepackage[utf8]{inputenc}
\usepackage{amsmath}
\usepackage{xcolor}
\usepackage{t1enc}
\usepackage{tikz}
\usepackage{graphicx}
\usetikzlibrary{arrows}
\usetikzlibrary{decorations.markings}
\usepackage[dvipsnames]{xcolor}
\usepackage{amssymb}
\usepackage{ragged2e}
\color{white}
\definecolor{Velvety Red}{RGB}{124,10,2}
\definecolor{Chalkboard Green}{RGB}{0,66,37}
\definecolor{Miscellaneous Blue}{RGB}{0,33,71}
\usepackage{graphicx}
\graphicspath{ {./images/} }


\begin{document}
\pagecolor{Miscellaneous Blue}
\color{white}
\Huge
\begin{center}
Simple Harmonic Motion
\end{center}
\begin{equation}
    x = a\sin{(\omega T)} \nonumber
\end{equation}
\begin{equation}
    x = a\cos{(\omega T)} \nonumber
\end{equation}
\newline\newline
\begin{center}
 Swipe For Derivation\end{center}
 \begin{center}
  $\Rightarrow$       
 \end{center}


\newpage
\normalsize
\noindent Below is an illustration that shows an object starting at a position of maximum amplitude.\newline
\begin{center}
 \includegraphics[width = 5cm, height = 10cm, angle = 90]{WhatsApp Image 2020-06-13 at 2.13.49 PM.jpeg}   
\end{center}
when we define the amplitude or displacement from the equilibrium position using the circular approach we get: 
\begin{equation}
x =  a\cos{(\theta)}   
\end{equation}


\noindent Now as we are using uniform circular motion to describe the simple harmonic motion of the object we can represent theta in terms of angular velocity and time period.
\begin{equation}
    \omega = \frac{\theta}{T} \to
    \omega T = \theta
\end{equation}
substituting (2) in (1) gives us:
\begin{equation}
    x = a\cos{(\omega T)}\nonumber
\end{equation}
Now if the object started it's motion from mean position at time t=0, we can use the cosine wave and move it $\frac{\pi}{2}$ radians or 90 degrees to the right hand side giving us or in other words you are simply taking theta to be the other angle and the variable x is just displacement from the center; we are just defining it in two different ways:
\begin{equation}
    x = a\cos{(\omega T-\frac{\pi}{2})}
\end{equation}
Since $\sin{(anything)} = \cos{(anything-\frac{\pi}{2})}$, we can rewrite (3) as:
\begin{equation}
    x = a\sin{(\omega T)}\nonumber
\end{equation}
Integrating these equations with respect to time gives you equations for velocity and if you integrate them twice then you get expressions for acceleration.
Pretty neat isn't it?
\end{document}
